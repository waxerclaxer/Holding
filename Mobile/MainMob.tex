

\documentclass{report}



\begin{document}
\title{The Possibilities in the Collection and Recording of Data For the Study and Improvement of Sky Diving and Formation forming}
\author{Michael Mason}

\maketitle

\begin{abstract}
When skydiving the ability to record data of the jump is essential when jumping in formation. With the use of inertial measurement units (IMU) it is possible to gain a better understanding of the technique used and improve on form. With the use of sensing equipment such as ; Magnetometer, accelerometer, gyroscope it is possible to get accurate results of pitch, yaw and roll. The choice of a good micro-controller will determine the over all efficiency.
Storing this data and being able to access it either between jumps or after a day of jumping is essential. Making this easy for the user to attain will be looked at in the form of SD cards and direct link. Blue-tooth will also be discussed.
Making the device energy efficient and possibilities of scavenging energy from wind and sun will be discussed. During the project, the price of materials will be in the forefront of thought, ensuring that a good price to efficiency ratio will be met.

\end{abstract}

\tableofcontents

\newpage

\section{Introduction}


\chapter{Hardware Selection and Design}
Choosing the correct hardware will define the overall practicality of the project. Several factors have to be taken into consideration when selecting hardware such as functionality, connections, power consumption and size with many more. These will be discussed in detail and compared to other products with indication why they were chosen over them. What storage is used for the device and the format in which it is stored in will be discussed
Both prototyping and final design will be discussed in separate stages to provide a structured design to the final product.
The three main sensing components tha will be used will be a gyroscope, accelerometer and a magnetometer.

\section{Breakout Boards}
For the prototyping breakout board will be used. By creating the prototype using breakout boards
 
\subsection{Micro-controller}

\subsection{Gyroscope}
The gyroscope that will be used in the prototype will be the Triple-Axis Digital-Output Gyro ITG-3200 breakout board. By using the inter-integrated 
  This particular model has been chosen die to the completeness of the device and it flexible power consumption rate 
\subsection{Accelerometer}






 
\end{document}

