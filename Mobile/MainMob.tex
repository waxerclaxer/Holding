
\documentclass{report}



\begin{document}
\title{Collection and Recording of Data For the Study and Improvement of Sky Diving and Formation forming}
\author{Michael Mason}

\maketitle
\begin{abstract}
When skydiving the ability to record data of the jump is essential when jumping in formation. With the use of inertial measurement units (IMU) it is possible to gain a better understanding of the technique used and improve on form. With the use of sensing equipment such as ; Magnetometer, accelerometer, gyroscope it is possible to get accurate results of pitch, yaw and roll. The choice of a good micro-controller will determine the over all efficiency.
Storing this data and being able to access it either between jumps or after a day of jumping is essential. Making this easy for the user to attain will be looked at in the form of SD cards and direct link. Blue-tooth will also be discussed.
Making the device energy efficient and possibilities of scavenging energy from wind and sun will be discussed. During the project, the price of materials will be in the forefront of thought, ensuring that a good price to efficiency ratio will be met.

\end{abstract}

\tableofcontents

\newpage

\section{Introduction}


\chapter{Hardware Selection and Design}
Choosing the correct hardware will define the overall practicality of the project. Several factors have to be taken into consideration when selecting hardware such as functionality, connections, power consumption and size with many more. These will be discussed in detail and compared to other products with indication why they were chosen over them. What storage is used for the device and the format in which it is stored in will be discussed
Both prototyping and final design will be discussed in separate stages to provide a structured design to the final product.
The three main sensing components that will be used will be a gyroscope, accelerometer and a magnetometer. Using all three allows for greater results. When combined together the accelerometer and gyroscope will work with each other to provide the pitch and roll. To receive the yaw result the manometer will be used.

\section{Breakout Board Prototype}
For the prototyping breakout board will be used. By creating the prototype using breakout boards it is possible to have a more practical and faster way to make the product. It will also be good to study the way that they are put together to then efficiently create the final product. A great advantage of using break out for the prototyping is that they include any resistors or capacitors that may be needed to reduce noise on the device. This will be considered when creating the final design, hopefully based on some ideas that the break out boards provide.
 
\subsection{Micro-controller}
The micro-controller is chosen due to the sensors that are needed. Available EEPROM memory is taken in to account to store system files. The micro controller that will be used for the prototype is PIC16F818. 
\begin{itemize}
\item 128 bytes of EEPROM data memory
\item 2 to 5.5 voltage
\item $i^2c$
\end{itemize}

\subsection{Gyroscope $i^2c$ connector}
The gyroscope that will be used in the prototype will be the Triple-Axis Digital-Output Gyro ITG-3200 breakout board. By using the inter-integrated 
This particular model has been chosen due to the completeness of the device and it flexible power consumption rate.
\begin{itemize}
\item voltage range of 2.1V to 3.6V
\item Fast Mode $I^2C$ (400kHz) serial interface
\end{itemize}
\subsection{Accelerometer $i^2c$ connector}
The triple axis MMA8452Q break out board will be sourced for the prototype. One of the great features of this particular device is the ability to stay in low power mode until told otherwise. This will save power and can be incorporated with possibly another embedded device such as the barometer or a simple switch to swap between power modes.
\begin{itemize}
\item ±2g/±4g/±8g dynamically selectable full-scale
\item I2C digital output interface
\item 1.6 V to 3.6 V interface voltage
\end{itemize}

\subsection{Magnetometer $i^2c$ connector}
Triple Axis Magnetometer Breakout - MAG3110 will be sourced for the magnetometer. This particular board comes with its own voltage regulator opposed to its main competitor, HMC5883L, which does not. This allows for a wider spectrum of voltage control .
\begin{itemize}
\item 1.95V to 3.6V Supply Voltage
\item 7-bit I2C address = 0x0E
\end{itemize}

\subsection{Barometer $i^2c$ connector}
Barometric Pressure Sensor - BMP180 Breakout will be used to measure pressure. This will be used to measure the altitude of the device. 
\begin{itemize}
\item Digital two wire (I²C, TWI, "Wire")
\item Ultra-low power consumption - Flexible supply voltage range (1.8V to 3.6V)
\end{itemize}

\subsection{SD Card Reader}
SD cards are non volatile storage. This allows information to be stored even when the device is not under direct power. The ability to remove the storage device will allow the user to retrieve the data collected and upgrade the devices storage. The component used for prototyping will be Breakout Board for microSD Transflash. This product will also be sourced out for the final design.
To 

\subsection{Power}
The aim goal for the power is small and rechargeable. With this in mind the ANSMANN 3.7V Wire lead terminal has been chosen.
\begin{itemize}
\item 3.7v nominal
\item 2250mAh
\item Lithium-Ion
\end{itemize}



\section{Final Product Specification}

\subsection{Micro-controller}
The micro controller that will be used for the prototype is PIC16F818. This is the same micro controller that was used in the prototype. For its capabilities and its price it seem the most efficient. With having 128Bytes EEPROM memory it is possible to store the code for the device on here. It will then export the information that is collected straight of the micro sd card ready for the user to transport to a computer for retrieval. The micro-controller also has bi-directional I/O ports which allow for multiple $I2^C$ connections.
\begin{itemize}
\item 128 bytes of EEPROM data memory
\item 2 to 5.5 voltage
\item $i^2c$
\end{itemize}
\begin{center}
  \begin{tabular}{ |l | c | r |}
    \hline
    Amount & Price W/O VAT £ & Price £ \\ \hline
    1 & 1.58 & 1.90 \\ \hline
    10 & 1.33 & 1.60 \\ \hline
    100 & 1.22 & 1.46   \\ \hline
  \end{tabular}
\end{center}

\subsection{Gyroscope and Accelerometer}
STMICROELECTRONICS - LSM330DLC functionas as both gyroscope and accelerometer. As opposed to the prototype model this unit contains both. The advantage of this is the connections used will be less. It is also by far and large the cheapest that has been sought out and works well with the voltages needed and the micro controller. It does not have any extra features such as stand by mode or a sleep mode.
\begin{itemize}
\item 2.4 - 3.6 voltage 
\item 
\end{itemize}




\subsection{Magnetometer}




\chapter{Control Algorithm and Software}


 
\end{document}

